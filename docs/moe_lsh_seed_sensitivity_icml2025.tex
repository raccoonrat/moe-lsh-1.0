%%%%%%%% ICML 2025 PAPER: Seed Sensitivity in MoE Semantic Watermarking %%%%%%%%%%%%%%%%%

\documentclass{article}

% Encoding and Chinese support for pdfLaTeX
\usepackage[utf8]{inputenc}
\usepackage{CJKutf8}

% Recommended packages for figures and better typesetting:
\usepackage{microtype}
\usepackage{graphicx}
\usepackage{subfigure}
\usepackage{booktabs} % for professional tables

% hyperref makes hyperlinks in the resulting PDF.
\usepackage{hyperref}

% Attempt to make hyperref and algorithmic work together better:
\newcommand{\theHalgorithm}{\arabic{algorithm}}

% Use the following line for the initial blind version submitted for review:
\usepackage{icml2025}

% For theorems and such
\usepackage{amsmath}
\usepackage{amssymb}
\usepackage{mathtools}
\usepackage{amsthm}

% if you use cleveref..
\usepackage[capitalize,noabbrev]{cleveref}

%%%%%%%%%%%%%%%%%%%%%%%%%%%%%%%%
% THEOREMS
%%%%%%%%%%%%%%%%%%%%%%%%%%%%%%%%
\theoremstyle{plain}
\newtheorem{theorem}{Theorem}[section]
\newtheorem{proposition}[theorem]{Proposition}
\newtheorem{lemma}[theorem]{Lemma}
\newtheorem{corollary}[theorem]{Corollary}
\theoremstyle{definition}
\newtheorem{definition}[theorem]{Definition}
\newtheorem{assumption}[theorem]{Assumption}
\theoremstyle{remark}
\newtheorem{remark}[theorem]{Remark}

% Short title for running head
\icmltitlerunning{Seed Sensitivity in MoE Semantic Watermarking}

\begin{document}
\begin{CJK}{UTF8}{gbsn}

\twocolumn[
\icmltitle{High-Dimensional Geometric Analysis of Seed Sensitivity in MoE Semantic Watermarking}

% Author information (blind for review)
\begin{icmlauthorlist}
\icmlauthor{Anonymous Author}{aff1}
\end{icmlauthorlist}

\icmlaffiliation{aff1}{Anonymous Institution}

\icmlkeywords{Machine Learning, Watermarking, Mixture of Experts, LSH, High-dimensional Geometry}

\vskip 0.3in
]

\printAffiliationsAndNotice{}

% Chinese display title (kept out of PDF bookmarks for hyperref)
\begin{center}
{\LARGE 混合专家模型与语义哈希水印中的随机种子敏感性}\\[0.5em]
\end{center}

\begin{abstract}
基于混合专家模型(Mixture-of-Experts, MoE)的 大语言模型(LLM)已在工业界和学术界广泛部署,水印技术被视为区分机器生成文本与人类创作文本的关键工具。
以 SemHash 为代表的语义水印方法利用局部敏感哈希(Locality-Sensitive Hashing, LSH)在嵌入空间中构造“红/绿”区域,在理论上具有较强的改写鲁棒性。
然而,实际系统表明此类方法在 MoE LLM 上呈现出极端的 \emph{随机种子敏感性}:极少数种子可达到 SOTA 生成质量和检测性能,而大部分随机初始化会导致困惑度爆炸或检测失效。
与其将种子视作单纯超参数,本文从高维几何和流形学习的角度系统刻画这一现象。
我们指出:LLM 嵌入空间呈现强烈的各向异性(“锥体效应”),而标准 LSH 预设的各向同性假设在此完全失效;随机超平面在高维各向异性锥体上极大概率出现“区域坍塌”,要么将整个语义簇标记为绿、无法注入熵,要么整体标记为红、迫使模型生成语义崩溃的 Token。
在 MoE 架构下,水印诱导的语义偏移进一步干扰专家路由,放大了种子敏感性。
为刻画和缓解该问题,我们提出一套基于语义簇分割熵、Logits 分布 Wasserstein 距离与 PCA 对齐度的“几何质量”指标,并给出白化变换、PCA 对齐 LSH 与基于聚类的非线性划分等数据依赖方案。
本文的结论是:种子敏感性不是偶然噪声,而是“各向异性流形 + 各向同性投影”几何错配的必然结果,解决路径应从被动选种转向主动重构几何和投影方向。
\end{abstract}

\section{引言:语义水印的几何不稳定性}
\label{sec:intro}

大语言模型(LLM)的生成能力快速提升,使得内容溯源与版权保护问题日益突出。
水印(Watermarking)通过在生成过程中注入可检测但难以察觉的统计信号,被视为当前最具可行性的治理手段之一。
经典的 Token 级水印(如 green-list 方案)通过伪随机函数将词表划分为“绿表/红表”,在生成时对绿表 Token 加性偏置,并在检测阶段进行统计显著性检验。
该类方法计算开销低、训练开箱即用,却对保持语义不变的改写攻击(Paraphrase Attack)极为脆弱。

为应对这一缺陷,近期工作转向 \emph{语义级水印}。
典型地,SemStamp/SemHash 通过外部句向量编码器将文本映射到 $ \mathbb{R}^d $ 中连续语义嵌入,再使用 LSH 将连续空间量化为离散哈希桶,用以驱动绿表选择。
在理想化模型中,若语义嵌入在单位超球上各向同性分布,则随机超平面可以稳定地按角度划分语义邻域,从而在语义邻近 Token 之间实现“低损失替换”。

\paragraph{MoE 场景下的工程痛点。}
当上述方法被部署到 MoE LLM(如 Mixtral、DeepSeek MoE、Qianwen MoE)时,实践中暴露出一个严重问题:
\emph{水印性能对随机种子(决定 LSH 超平面)呈现极端敏感性}。
少数“幸运种子”可以在几乎不损伤生成质量的前提下实现高检测率,而大部分随机种子要么几乎检测不到水印,要么显著拉高困惑度、产生语义错乱输出。

本文尝试回答:这种种子敏感性是偶然现象,还是高维几何结构的必然结果?
如果是后者,我们是否可以在训练前、部署前对种子进行几何质量评估与筛选,甚至通过重新设计哈希空间来消除此类敏感性?

\subsection{MoE 架构下的特殊性:路由几何的放大效应}
\label{subsec:moe-specific}

混合专家模型(MoE)在每一层包含多个专家网络和一个路由门控网络。
对于给定 Token 隐状态 $ \mathbf{h}_t $,路由器计算专家权重 $ \mathbf{r}_t \in \mathbb{R}^E $ 并选择 top-$k$ 专家进行激活。
该机制一方面带来参数高效扩展,另一方面引入了高度结构化的几何边界:路由决策本质上由一系列高维超平面刻画。

当在 MoE 上叠加 SemHash 风格的水印平面时,系统中同时存在两套几何划分:
\begin{itemize}
    \item 内生的 \emph{路由平面},决定哪些专家被激活;
    \item 外加的 \emph{水印平面},决定哪些 Token 被视为“绿表候选”。
\end{itemize}
若水印平面在语义空间中强行抑制了原本高概率的候选 Token,将导致 $ \mathbf{h}_t $ 发生非局部偏移,使得路由器激活与上下文语义不匹配的专家。
我们将这种现象称为 \emph{路由错位(Routing Misalignment)}。
在长上下文生成中,这种错位具有级联效应:一处水印诱导的错误选择可在后续层层放大,导致整体语义漂移。

因此,MoE 场景下的种子敏感性不仅反映在单步 Logits 分布被扰动,更体现在路由轨迹这一复杂几何对象被扭曲。

\subsection{本文贡献}
\label{subsec:contrib}

围绕上述现象,本文从高维几何与流形学习的角度提出以下贡献:
\begin{enumerate}
    \item \textbf{几何机理刻画:} 揭示 LLM 嵌入空间中“锥体效应”与 LSH 各向同性假设之间的根本矛盾,给出随机超平面在各向异性锥体上高概率导致“区域坍塌”的几何推导。
    \item \textbf{语义碎片化分析:} 建立语义聚类与 Logits 选择的几何模型,解释糟糕种子如何通过切断高概率语义簇而迫使模型选择语义噪声 Token。
    \item \textbf{指标体系设计:} 提出三类可在部署前计算的种子质量指标:语义簇分割熵、Logits 分布的 Wasserstein 距离以及投影方差/PCA 对齐度。
    \item \textbf{几何修正方案:} 讨论白化变换、PCA 对齐 LSH 以及基于质心的非线性划分作为缓解种子敏感性的技术路径。
\end{enumerate}

\section{高维语义空间的几何特性}
\label{sec:geometry}

\subsection{理想各向同性与现实各向异性}
\label{subsec:isotropy}

经典 LSH 理论多基于如下假设:数据点 $x \in \mathbb{R}^d$ 在单位超球 $ \mathbb{S}^{d-1}$ 上各向同性分布。
以 SimHash 为例,在该假设下,两向量夹角 $\theta(x,y)$ 与哈希碰撞概率近似线性关系:
\begin{equation}
    \mathbb{P}[h(x)=h(y)] = 1 - \frac{\theta(x,y)}{\pi}.
\end{equation}
随机投影向量 $r \sim \mathcal{N}(0, I_d)$ 被视为在超球上均匀采样,其对应的超平面以“公平”的方式切割数据。

然而,大量实证工作表明:BERT、GPT、Mixtral 等模型的上下文嵌入远非各向同性,而是呈现显著的 \emph{各向异性(Anisotropy)}。
记嵌入矩阵为 $X \in \mathbb{R}^{N \times d}$,协方差为
\begin{equation}
    \Sigma = \frac{1}{N} X^\top X.
\end{equation}
在各向异性场景下,特征值谱表现为极少数主成分 $\lambda_1, \lambda_2, \ldots$ 占据绝大多数能量,其余特征值快速衰减。
这意味着数据实际集中在一个远低于 $d$ 维的流形上,并且该流形具有显著朝向,可视为被“压缩”在一个开口角极小的锥体 $\mathcal{K}$ 内。

更直观地,对任意两个随机词向量 $u,v$,经验上有
\begin{equation}
    \mathbb{E}[\cos(u,v)] \approx 1 - \Delta, \quad \Delta \ll 1,
\end{equation}
与各向同性高维空间“随机向量近似正交”的结论相反。

\subsection{随机投影在锥体分布下的失效}
\label{subsec:cone-failure}

考虑法向量 $r \sim \mathcal{N}(0, I_d)$ 对应的超平面 $H_r$ 与数据锥体 $\mathcal{K}$ 的相互位置。
根据高维测度集中的经典结果,$r$ 与锥体主轴方向 $\mu$ 的夹角 $\phi$ 高度集中于 $\pi/2$ 附近,即大部分随机方向与 $\mu$ 近似正交。

然而,要让 $H_r$ 有效地对数据产生“划分”作用,平面必须穿过锥体内部;否则所有样本将全部落在同侧。
若锥体张角为 $\alpha \ll 1$,则只有当 $r$ 落入一个极窄的“赤道带”时,平面才会与锥体发生非平凡交集。
由此可见:
\begin{itemize}
    \item 绝大多数随机种子对应的平面 \emph{完全掠过锥体},导致锥体内所有点同侧。
    \item 少数“幸运种子”恰好对应穿过锥体的平面,才能实现有效二分。
\end{itemize}

\paragraph{区域坍塌(Region Collapse)。}
在极端各向异性下,随机平面对数据的典型行为可概括为两种情形:
\begin{description}
    \item[情形 A:失效] 超平面完全位于锥体一侧。此时所有语义合理的候选 Token 被统一标记为绿或红:
    \begin{itemize}
        \item 若全为绿,则水印对生成无约束,检测信号消失;
        \item 若全为红,则模型被迫在语义无关的尾部分布中寻找“可用” Token,生成质量崩溃。
    \end{itemize}
    \item[情形 B:有效] 超平面侥幸穿过锥体,在局部实现了非平凡切割,从而允许在语义邻近 Token 间实施偏置。
\end{description}
MoE 场景下实验观察到的“好种子极少、坏种子占绝大多数”,正是情形 B 占比随维度与各向异性程度指数级下降的体现。

\section{语义碎片化与 Logits 选择}
\label{sec:fragmentation}

用户关注的第一个问题是:糟糕的种子如何从几何上破坏语义聚类,并在 Logits 层面推动模型选择语义噪声 Token。

\subsection{语义聚类的几何刻画}
\label{subsec:cluster-model}

在 next-token 预测中,给定上下文 $C$,模型输出 Logits 向量 $\mathbf{z}$,高概率 Token 往往集中在一个或多个紧密簇中。
例如对于“The cat sat on the ...”,最高概率 Token 集合
\begin{equation}
    \mathcal{T}_{\text{top}} = \{\text{mat}, \text{rug}, \text{floor}, \text{sofa}\}
\end{equation}
在嵌入空间中形成一个半径极小的超球 $\mathcal{B}_\epsilon(c)$,其中 $c$ 为簇中心、$\epsilon$ 很小。

\subsection{碎片化切割与强制降级}
\label{subsec:fragmentation-bad}

设由种子 $S$ 决定的超平面 $H_S$ 将空间划分为水印“绿区” $V_{\text{G}}$ 与“红区” $V_{\text{R}}$。
理想情况下,$H_S$ 对每个语义簇都实现近似 $50\%/50\%$ 切分,使得至少存在若干同义词落在绿区中,模型可在低语义代价下替换 Token。
但在锥体背景与随机投影机制下,更典型的却是以下两种极端:

\begin{itemize}
    \item 整个高质量簇 $ \mathcal{B}_\epsilon(c)$ 被压入红区,绿区只包含语义无关或低概率 Token;
    \item 整个簇进入绿区,无法注入任何水印信息。
\end{itemize}

记最优 Token 为 $t^*$,其 Logit 为 $z_{\max}$,次优同义词为 $t'$,Logit 为 $z_{\text{sub}} \approx z_{\max}$。
对绿区 Token 施加偏置 $\delta>0$ 后,生成概率为
\begin{equation}
    P(t) \propto \exp\left(z_t + \delta \cdot \mathbb{I}[t\in V_{\text{G}}]\right).
\end{equation}
当 $t^* \in V_{\text{R}}$ 且所有高质量同义词均被切到红区时,模型被迫在 $V_{\text{G}}$ 内选择 Logit 最大的 $t_{\text{noise}}$,其原始 Logit $z_{\text{noise}} \ll z_{\max}$。
若
\begin{equation}
    z_{\text{noise}} + \delta > z_{\max},
\end{equation}
则模型输出 $t_{\text{noise}}$,表现为明显的语义漂移甚至胡言乱语。
我们将此过程称为 \emph{强制降级(Forced Downgrade)}。

\subsection{低熵任务中的区域坍塌}
\label{subsec:low-entropy}

在低熵生成任务(例如结构化摘要、事实问答)中,模型的高质量输出空间往往近似退化为一个极小邻域甚至“点”。
当 $ \epsilon \to 0$ 时,任意超平面对该“点簇”的切割要么是“全进全出”,要么几乎不产生有效边界。
此时 SemHash 试图在该簇内部注入熵的企图注定失败:
要么完全无法检测,要么在某一关键步整体封锁高质量空间,引发灾难性生成错误。

\section{各向异性与随机投影的低效性}
\label{sec:anisotropy}

\subsection{投影方差的极度不均衡}
\label{subsec:proj-var}

LSH 的区分能力依赖于投影值
\begin{equation}
    y = r^\top x
\end{equation}
在数据分布上的方差:
\begin{equation}
    \mathrm{Var}(y) = r^\top \Sigma r.
\end{equation}
在强各向异性下,$\Sigma$ 的能量集中在前 $k$ 个主成分 $u_1,\dots,u_k$ 上。
高维几何告诉我们:对随机向量 $r$,其在任一固定方向上的投影 $ \langle r, u_i \rangle $ 期望极小。
因此,对绝大多数种子而言,$ \mathrm{Var}(y)$ 接近 $0$,即“切了空气而非数据”。

类比三维“黄瓜”:
有效的水印平面应沿长轴方向切割,而随机平面大概率平行于长轴且远离黄瓜本体,或仅掠过其边缘。

\subsection{碰撞概率退化与辨识度丧失}
\label{subsec:collision-degeneration}

在各向异性锥体中,任意两个有效语义向量 $x_1,x_2$ 的夹角 $\theta$ 极小(例如 $\theta < 15^\circ$)。
由 SimHash 碰撞关系可得
\begin{equation}
    \mathbb{P}[h(x_1)=h(x_2)] = 1 - \frac{\theta}{\pi} \approx 1,
\end{equation}
意味着对绝大部分种子,所有语义相关向量被哈希到相同桶中。
这带来两方面后果:
\begin{itemize}
    \item \textbf{辨识度丧失}:水印难以在不同语义状态之间制造差异;
    \item \textbf{鲁棒性两极化}:若该桶被标记为绿,则极难检测;若被标记为红,则极易毁坏生成。
\end{itemize}
因此,在 LLM 的各向异性嵌入空间中,\emph{未经修正的随机投影在理论上注定效率低下且不稳定}。

\section{量化种子几何质量的指标体系}
\label{sec:metrics}

为了在部署前筛选种子,我们提出三类互补的“几何质量”指标。

\subsection{语义簇分割熵:刻画碎片化程度}
\label{subsec:split-entropy}

首先在一个校准数据集上,用无水印模型构造词表语义聚类:
对 Token 嵌入做 k-means,得到 $K$ 个簇 $C_1,\dots,C_K$。
对给定种子,计算全词表红绿划分 $V_{\text{G}}, V_{\text{R}}$。
对每个簇 $C_i$,定义绿表比例
\begin{equation}
    r_i = \frac{|C_i \cap V_{\text{G}}|}{|C_i|}.
\end{equation}
理想情况下 $r_i \approx 0.5$。
我们定义分割熵得分为
\begin{equation}
    \mathrm{Score}_{\text{split}}
    = 1 - \frac{1}{K} \sum_{i=1}^{K} \left(2|r_i - 0.5|\right)^2.
\end{equation}
当所有簇被整体压入绿表或红表时,$r_i \in \{0,1\}$,得分趋近 0,意味着严重区域坍塌;得分越接近 1,说明切分越均匀,语义碎片化风险越低。

\subsection{Logits 分布的 Wasserstein 距离:刻画“推土机距离”}
\label{subsec:wasserstein}

考虑单步生成中,原始 Softmax 分布 $P$ 与施加水印偏置后的分布 $Q$。
令 $D_{ij} = \| \mathrm{emb}(i) - \mathrm{emb}(j)\|_2$ 表示 Token 之间的语义距离矩阵。
则 Wasserstein-1 距离为
\begin{equation}
    W_1(P,Q) = \inf_{\gamma \in \Pi(P,Q)} \mathbb{E}_{(x,y)\sim \gamma}[\|x-y\|_2],
\end{equation}
其中 $\Pi(P,Q)$ 为所有以 $P$ 与 $Q$ 为边缘分布的联合分布集合。
直观上,$W_1$ 度量“将概率质量从 $P$ 推到 $Q$ 需要搬运的语义距离总量”。

\begin{itemize}
    \item 好的种子使 $W_1$ 保持较小,仅在同义词等近邻 Token 间重新分配概率;
    \item 坏种子导致 $W_1$ 显著增大,说明概率质量被迫迁移到语义遥远的 Token 上。
\end{itemize}

在实践中,可在代表性任务与步长上估计 $W_1$ 的期望或分位数,将其作为种子筛选的硬约束。

\subsection{投影方差与 PCA 对齐度:刻画各向异性适应度}
\label{subsec:pca-alignment}

在校准集上收集嵌入矩阵 $X$,对给定种子对应的投影向量 $r$,计算
\begin{equation}
    h = X r, \quad \sigma^2_{\text{proj}} = \mathrm{Var}(h).
\end{equation}
若 $ \sigma^2_{\text{proj}}$ 极小,说明 $r$ 近似垂直于数据流形,平面主要切到“空旷区域”;反之,较大的方差意味着平面沿着数据主要变化方向切割,更有可能穿过锥体核心。

进一步,可利用 PCA 主成分 $u_1,\dots,u_k$ 定义
\begin{equation}
    \mathrm{Align}(r) = \sum_{i=1}^{k} \cos^2(r, u_i),
\end{equation}
该值越大,说明种子越“对齐”于高能量方向,从几何上更有希望获得稳定且高熵的哈希划分。

\section{解决种子敏感性的几何方案}
\label{sec:solutions}

基于前述分析,我们认为解决种子敏感性的关键不是“试出一个好种子”,而是重构数据几何或投影机制,使\emph{大部分种子都变得可用}。

\subsection{白化变换:修正嵌入几何}
\label{subsec:whitening}

在进行 LSH 之前,对嵌入向量进行白化:
\begin{equation}
    \tilde{x} = W(x-\mu), \quad W = \Sigma^{-1/2},
\end{equation}
其中 $\mu$ 与 $\Sigma$ 分别为均值与协方差估计。
白化操作将原本高度各向异性的锥体拉伸为近似各向同性的球体。
在此新空间中,随机投影重新符合经典 LSH 假设,使种子敏感性大幅降低。
代价主要为一次性离线估计 $W$ 和在线前向中的轻量线性变换。

\subsection{PCA 对齐的 LSH:修正投影方向}
\label{subsec:pca-lsh}

另一条路径是放弃完全均匀的随机 $r$,而从数据主成分中抽取投影向量。
具体地,在校准集上对嵌入做 PCA,取前 $k$ 个主成分 $u_1,\dots,u_k$ 作为 LSH 的法向量族。
这种 PCA-LSH 方案保证每个超平面都沿着数据方差最大的方向切割,从而必然与锥体核心发生交集,避免“切空气”。

在此基础上,可对主成分进一步做随机旋转或正交变换,以在保证高方差的前提下引入足够的密钥空间。

\subsection{基于质心的非线性划分:拥抱数据拓扑}
\label{subsec:centroid}

更激进的策略是完全放弃线性超平面,将水印划分建立在数据驱动的聚类结构上。
例如 k-SemStamp 提出:
\begin{enumerate}
    \item 对语义嵌入做 k-means 聚类,获得簇中心 $\{c_j\}_{j=1}^{K}$;
    \item 使用 Voronoi 划分定义语义单元,再随机给簇分配“红/绿”标签。
\end{enumerate}
由于聚类本身追求簇内紧致性、簇间分离度,该方案天然避免了在簇内部“硬切割”,从根本上消解语义碎片化问题。
其缺点是需要离线聚类和在线最近质心查询。

\section{结论}
\label{sec:conclusion}

本文从高维几何和流形学习的视角,对 MoE LLM 中基于 SemHash 的语义水印所呈现的极端随机种子敏感性进行了系统分析。
我们的结论可以概括为:
\begin{itemize}
    \item LLM 嵌入空间的“锥体效应”与 LSH 的各向同性假设存在根本冲突,使得随机超平面在高维中高概率导致“区域坍塌”,要么无效、要么毁灭性破坏生成;
    \item 语义碎片化源于超平面正交于语义流形主轴,切断高概率语义簇并强制模型在尾部分布中选取噪声 Token,在 MoE 场景下还会诱导路由错位;
    \item 通过语义簇分割熵、Wasserstein 距离与投影方差/PCA 对齐度等指标,可以在部署前对种子几何质量进行定量评估;
    \item 真正可行的解决路径应从“试运气选种”转向“重构几何与投影”:通过嵌入白化、PCA 对齐 LSH 或基于聚类的非线性划分,使大多数种子在数学上都具有稳定且可解释的行为。
\end{itemize}

从更宏观的角度看,MoE 水印中的种子敏感性问题并非孤立现象,而是“低维流形数据 + 高维随机机制”组合下常见的几何陷阱。
我们希望本文的分析能够为后续设计鲁棒、可解释且几何自洽的水印方案提供理论基线。

\section*{致谢}

本工作在匿名评审阶段暂不列出具体致谢对象。

\section*{影响声明}

本研究聚焦于面向大语言模型的水印技术,其潜在影响具有双刃性。
一方面,鲁棒水印有助于内容溯源、版权保护与错误信息检测;另一方面,不当使用也可能带来隐私、审查与滥用风险。
我们鼓励在公开透明的前提下部署水印系统,结合法律与伦理框架,避免将技术用于牺牲用户自主权的场景。

\bibliography{moe_lsh_watermark_refs}
\bibliographystyle{icml2025}

\end{CJK}
\end{document}


